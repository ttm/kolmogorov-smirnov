\begin{table}[h!]
\begin{center}
\begin{tabular}{| l | c | c | c | c | c | c |}\hline
$\alpha N_c$ & $\alpha$ & $c(\alpha)$ & $|C_1(\alpha)|$ & $|C_2(\alpha)|$ & $|C_3(\alpha)|$ & $|C_4(\alpha)|$ \\\hline
100.0 & 0.100 & 1.22 & 0 & 87 & 69 & 73 \\\hline
50.0 & 0.050 & 1.36 & 0 & 47 & 31 & 25 \\\hline
25.0 & 0.025 & 1.48 & 0 & 19 & 17 & 13 \\\hline
10.0 & 0.010 & 1.63 & 0 & 7 & 6 & 4 \\\hline
5.0 & 0.005 & 1.73 & 0 & 3 & 4 & 1 \\\hline
1.0 & 0.001 & 1.95 & 0 & 1 & 0 & 0 \\\hline
\end{tabular}
\caption{The theoretical maximum number $\alpha N_c$ of rejections
of the null hypothesis for critical values of $\alpha$.
The $c_1$ values were calculated using simulations of 1-parameter Weibull distributions with $a=0.1$.
The $c_2$ values were calculated using simulations of 1-parameter Weibull distributions with $a=2$.
The $c_3$ values were calculated using simulations of 1-parameter Weibull distributions with $a=4$.
Over all $N_c$ comparisons,
The $N_o$ values of $c_4$ were calculated using simulations of
 1-parameter Weibull distributions with $a=6$.
Over all $N_c$ comparisons,
 $\mu(c_1)=0.0289$ and $\sigma(c_1)=0.0179$,
 $\mu(c_2)=0.8306$ and $\sigma(c_2)=0.2660$,
 $\mu(c_3)=0.8001$ and $\sigma(c_3)=0.2520$ .
 $\mu(c_4)=0.8043$ and $\sigma(c_4)=0.2530$ .
}
\end{center}
\end{table}