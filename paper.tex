% ****** Start of file aipsamp.tex ******
%
%   This file is part of the AIP files in the AIP distribution for REVTeX 4.
%   Version 4.1 of REVTeX, October 2009
%
%   Copyright (c) 2009 American Institute of Physics.

% Use this file as a source of example code for your aip document.
% Use the file aiptemplate.tex as a template for your document.
\documentclass[%
	aip,
	jmp,%
	amsmath,amssymb,
	%preprint,%
	reprint,%
	%author-year,%
	%author-numerical,%
]{revtex4-1}
\usepackage{graphicx}% Include figure files
\usepackage{grffile}
\usepackage{dcolumn}% Align table columns on decimal point
\usepackage{bm}% bold math
%\usepackage[mathlines]{lineno}% Enable numbering of text and display math
%\linenumbers\relax % Commence numbering lines
\usepackage{multirow}
\usepackage{color} % for the notes
\usepackage{etex}
\reserveinserts{58}
%\usepackage{morefloats}
\usepackage{hyperref}
\usepackage{xcolor}
\usepackage{amsmath}
\hypersetup{
	colorlinks,
	linkcolor={red!50!black},
	citecolor={blue!50!black},
	urlcolor={blue!80!black}
}

\usepackage{xr}
\externaldocument{supportingInformation}
\maxdeadcycles=1000

\usepackage{placeins}
\begin{document}

\preprint{XXXXX (preprint)}

%\title[Evolution of interaction networks]{On the evolution of interaction networks: primitive typology of vertex, prominence of measures and activity statistics}% Force line breaks with \\
%\title[Evolution of interaction networks]{On the evolution of interaction networks: a primitive typology of vertex}% Force line breaks with \\
%\title[Stability of interaction networks]{Stability in human interaction networks: sector relative sizes, prominence of topological measures and time activity statistics.}% Force line breaks with \\
%\title[Stability in human interaction networks]{Sector relative sizes and topological metrics time stability in human interaction networks}% Force line breaks with \\
\title[Distance between histograms]{Distances between histograms}% Force line breaks with \\
\author{Renato Fabbri}%
\homepage{http://ifsc.usp.br/~fabbri/}
\email{fabbri@usp.br}
\affiliation{ 
	S\~ao Carlos Institute of Physics, University of S\~ao Paulo (IFSC/USP),
	PO Box 369, 13560-970, S\~ao Carlos, SP, Brazil %\\This line break forced with \textbackslash\textbackslash
}

\date{\today}% It is always \today, today,
%  but any date may be explicitly specified

\begin{abstract}
This document presents reference values for a distance metric
derived from the Kolmogorov-Smirnov statistical test.
Each measure is a distance between two histograms.
The sections are self-explanatory
on deriving benchmarks by comparing samples from usual distributions
and on exemplifying the power of the acquired knowledge.
\end{abstract}

\pacs{05.10-a,}% PACS, the Physics and Astronomy
\keywords{Kolmogorov-Smirnov test, benchmark, distance measure, histogram}
\maketitle
\section{Introduction}\label{sec:intro}
Be $F_{1,n}$ and $F_{2,n'}$ two empirical cumulative distributions,
where $n$ and $n'$ are the number of observations on each sample.
The two-sample Kolmogorov-Smirnov test rejects the null hypothesis
(that the histograms are the outcome of the same underlying distribution)
if:
\begin{equation}\label{eq:ks}
D_{n,n'} > c(\alpha)\sqrt{\frac{n+n'}{nn'}}
\end{equation}

\noindent where $D_{n,n'}=sup_x[F_{1,n}-F_{2,n'}]$ and $c(\alpha)$ are related to the critical region $\alpha$ by:

\begin{table}[h!]
\centering
\begin{tabular}{|l||c|c|c|c|c|c|}\hline
$\alpha$    & 0.1  & 0.05 & 0.025 & 0.01 & 0.005 & 0.001 \\\hline
$c(\alpha)$ & 1.22 & 1.36 & 1.48  & 1.63 & 1.73  & 1.95  \\\hline
\end{tabular}
\end{table}

If distributions are drawn from empirical data, $D_{n,n'}$ is given as are $n$ and $n'$.
All terms in equation~\ref{eq:ks} are positive and $c(\alpha)$ can be isolated:

\begin{equation}\label{eq:ks2}
c(\alpha) < \frac{D_{n,n'}}{\sqrt{\frac{n+n'}{nn'}}} = c'
\end{equation}

%Tables~\ref{tab:kolSub}-\ref{tab:kolPctInter} are populated with values for $c'(\alpha)$
When $c'$ is high, low values of $\alpha$ favor rejecting the null hypothesis.
For example, when $c'$ is greater than $\approx 1.7$,
one might assume that $F_{1,n}$ and $F_{2,n'}$
are outcomes of different distributions.
More importantly for us is that $c'$
is a measure of distance between both distributions~\cite{kolm}.
The main contribution of the following sections is the
explicit display of reference values from which
one might derive knowledge from collections of empirical measures of $c'$
or even of a single value of $c'$.

\subsection{Philosophical and technological note}
Difference and equivalence is of central role in human cognition,
philosophy and science.
This fact is so deeply recognized that thinkers often reduce
thought to classifications, e.g. through the
mathematical concept of equivalence classes~\cite{deleuze,physics}.
Histograms are very immediate and informative
wherever there is a phenomenon of interest which can yield measurements.
This present document should enable conclusions to be drawn about 
the equivalence (and difference)
of the processes underlying sets of measurements for a very
broad range of phenomena.

\subsection{Document outline}
Section~\ref{sec:simulations} expose reference values drawn from simulations.
Section~\ref{sec:empirical} exemplifies the use of such reference values
to make sense of phenomena.
Section~\ref{sec:conc} hold final remarks.
Software and data specification are given in Appendix~\ref{ap:soft}.


\section{References through simulations}\label{sec:simulations}
On every case, values of $c'$ are given for simulations involving
at least normal, uniform, triangular, Weibull and power-law distributions.

\subsection{When the null hypothesis is true}
If the null hypothesis is true, than the number
of rejections of the null hypothesis (that is: $c'>c(\alpha)$)
in $N_c$ comparisons should not exceed $\alpha N_c$.
To verify this, let $C'=\{c_i'\}$ be a set of $c'$ measures,
and $C'(\alpha)=\{c' : c'>c(\alpha)\}$.
Be $|C'(\alpha)|$ the cardinality of $C'(\alpha)$,
i.e. the number of comparisons in which the two-sample Kolmogorov-Smirnov
test rejects the null hypothesis for a given $\alpha$.

The most important result of this section is that
$|C'(\alpha)|$ rarely exceeds $\alpha N_c$,
 where $N_c$ is the number of comparisons made,
no matter what probability distribution type
or the specific setting.
Also important are that
$c'>c(\alpha)$ in many cases
and that $\alpha$ is a good estimate of the upper limit of the probability
of such an event.
%\begin{itemize}
%	\item $c'>c(\alpha)$ in many cases, and $\alpha N_c$ is a good upper limit to keep in mind.
%	\item 
%\end{itemize}

% Input tables
% any plot? If std is ~stable, plots of the mean of c~(\alpha) are compact and informative
\begin{table}[h!]
\begin{center}
\begin{tabular}{| l | c | c | c | c | c |}\hline
$\alpha N_c$ & $\alpha$ & $c(\alpha)$ & $|C_1(\alpha)|$ & $|C_2(\alpha)|$ & $|C_3(\alpha)|$ \\\hline
10.0 & 0.100 & 1.22 & 8 & 7 & 5 \\\hline
5.0 & 0.050 & 1.36 & 1 & 2 & 2 \\\hline
2.5 & 0.025 & 1.48 & 1 & 0 & 1 \\\hline
1.0 & 0.010 & 1.63 & 1 & 0 & 0 \\\hline
0.5 & 0.005 & 1.73 & 1 & 0 & 0 \\\hline
0.1 & 0.001 & 1.95 & 0 & 0 & 0 \\\hline
\end{tabular}
\caption{The theoretical maximum number $\alpha N_c$ of rejections
        of the null hypothesis for significance levels $\alpha$.
        The $c_1$ values were calculated using simulations of normal distributions with $\mu=0$ and $\sigma=1$.
        The $c_2$ values were calculated using simulations of normal distributions with $\mu=3$ and $\sigma=2$.
        The $c_3$ values were calculated using simulations of normal distributions with $\mu=6$ and $\sigma=3$.
        Over all $N_c$ comparisons,
         $\mu(c_1)=0.7775$ and $\sigma(c_1)=0.2573$,
         $\mu(c_2)=0.7976$ and $\sigma(c_2)=0.2388$,
         $\mu(c_3)=0.7596$ and $\sigma(c_3)=0.2292$ .
        }
\end{center}
\end{table}
\begin{table}[h!]
\begin{center}
\begin{tabular}{| l | c | c | c | c | c |}\hline
$\alpha N_c$ & $\alpha$ & $c(\alpha)$ & $|C_1(\alpha)|$ & $|C_2(\alpha)|$ & $|C_3(\alpha)|$ \\\hline
100.0 & 0.100 & 1.22 & 93 & 88 & 95 \\\hline
50.0 & 0.050 & 1.36 & 43 & 42 & 39 \\\hline
25.0 & 0.025 & 1.48 & 19 & 28 & 17 \\\hline
10.0 & 0.010 & 1.63 & 7 & 10 & 12 \\\hline
5.0 & 0.005 & 1.73 & 4 & 4 & 8 \\\hline
1.0 & 0.001 & 1.95 & 1 & 0 & 1 \\\hline
\end{tabular}
\caption{The theoretical maximum number $\alpha N_c$ of rejections
of the null hypothesis for critical values of $\alpha$.
The $c_1$ values were calculated using simulations of uniform distributions within $[0,1)$.
The $c_2$ values were calculated using simulations of uniform distributions within $[2,6)$.
The $c_3$ values were calculated using simulations of uniform distributions with $\mu=4$ and $\sigma=10$.
Over all $N_c$ comparisons,
 $\mu(c_1)=0.8374$ and $\sigma(c_1)=0.2642$,
 $\mu(c_2)=0.8377$ and $\sigma(c_2)=0.2623$,
 $\mu(c_3)=0.8406$ and $\sigma(c_3)=0.2696$ .
}
\end{center}
\end{table}
\begin{table}[h!]
\begin{center}
\begin{tabular}{| l | c | c | c | c | c | c |}\hline
$\alpha N_c$ & $\alpha$ & $c(\alpha)$ & $|C_1'(\alpha)|$ & $|C_2'(\alpha)|$ & $|C_3'(\alpha)|$ & $|C_4'(\alpha)|$ \\\hline
10.0 & 0.100 & 1.22 & 0 & 2 & 6 & 5 \\\hline
5.0 & 0.050 & 1.36 & 0 & 2 & 3 & 3 \\\hline
2.5 & 0.025 & 1.48 & 0 & 0 & 1 & 2 \\\hline
1.0 & 0.010 & 1.63 & 0 & 0 & 0 & 1 \\\hline
0.5 & 0.005 & 1.73 & 0 & 0 & 0 & 0 \\\hline
0.1 & 0.001 & 1.95 & 0 & 0 & 0 & 0 \\\hline
\end{tabular}
\caption{The theoretical maximum number $\alpha N_c$ of rejections
of the null hypothesis for critical values of $\alpha$.
The number of comparisons is $N_c=100$,
each with the sample size of $N_o=1000$ observations.
Each histogram have $N_b=30$ equally spaced bins.
The $N_o$ values of $c_1'$ were calculated using simulations of
 1-parameter Weibull distributions with $a=0.1$.
The $N_o$ values of $c_2'$ were calculated using simulations of
 1-parameter Weibull distributions with $a=2$.
The $N_o$ values of $c_3'$ were calculated using simulations of
 1-parameter Weibull distributions with $a=4$.
Over all $N_c$ comparisons,
The $N_o$ values of $c_4'$ were calculated using simulations of
 1-parameter Weibull distributions with $a=6$.
Over all $N_c$ comparisons,
 $\mu(c_1')=0.0622$ and $\sigma(c_1')=0.0397$,
 $\mu(c_2')=0.6934$ and $\sigma(c_2')=0.2186$,
 $\mu(c_3')=0.7638$ and $\sigma(c_3')=0.2684$ .
 $\mu(c_4')=0.7283$ and $\sigma(c_4')=0.2810$ .
}
\end{center}
\end{table}
% distribuicoes normal, uniforme, weibul
% distribuicao de lei de potencia
% contemplar numeros diferentes de bins, de amostras e de distribuicoes
% enquanto mantemos mudando os parametros
\section{References in empirical data}\label{sec:empirical}
% nltk com machado, shakespeare e biblia
% arquivo de audio
% bytes quaisquer de algum arquivo?
% dados puxados da wikipedia, gmane ou ?
\section{Conclusions}\label{sec:conc}
\begin{acknowledgments}
	Financial support was obtained from CNPq (140860/2013-4,
	project 870336/1997-5), United Nations Development Program (contract: 2013/000566; project BRA/12/018) and FAPESP. 
	The authors are grateful to the American Jewish Committee for maintaining an online copy of the Adorno book used on the epigraph~\cite{adorno}, to GMANE creators and maintainers for the public email list data, to the communities of the email lists and other groups used in the analysis, and to the Brazilian Presidency of the Republic for keeping Participabr code and data open.
	We are also grateful to developers and users of Python scientific tools.
\end{acknowledgments}


\appendix
\section{Software and data specifications}\label{ap:soft}


%%%%%%%%%%%%%%%%%%%%%%%%%%%%%%%%%%%%%%%
%\nocite{*}
\bibliography{paper}% Produces the bibliography via BibTeX.

\end{document}
